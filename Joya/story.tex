\documentclass{article}
\usepackage[utf8]{inputenc}
\usepackage{setspace}
\usepackage{graphicx} % Required for images

\title{\textbf{Story Writing For Class 10}}
\author{\textbf{Dr. Khastagir Government Girls' High School}}
\date{}

\begin{document}
\maketitle
\begin{center}
    \includegraphics[width=0.2\textwidth]{images.jpg}
\end{center}

\maketitle
\subsection*{\begin{center}\normalfont This document was made by Sakib\end{center}}

\section*{1.Perseverance is the Key to Success}
Once upon a time, there was a young girl named Fatima in a small village. She always dreamed of being a doctor but her father was too poor to bear her educational expenses. Despite being a daughter of a poor father, Fatima was determined to pursue her dream. Every day she woke up early in the morning and studied till going to school. Her school was very far. She went to school on foot. However, she worked hard and excelled in her studies. Accordingly, she obtained a golden A+ in the SSC examination and got admitted into a village college. As her father couldn’t afford her school fees, she found ways to earn the money herself. She sold homemade goods, earned some money, and maintained her educational expenses. As she grew older, she faced many problems. Many of her peers dropped out of school getting married. But she didn't lose her hope. She continued to work hard and persevere. Through difficulties, she also passed the HSC examination with a golden A+ and got admitted into a medical college. After getting admitted into the medical college, she earned a scholarship. She worked tirelessly and studied long hours to make a good result in the MBBS examination. She also did a part-time job. However, she never lost her goal because she knew that perseverance was the key to success. After years of hard work, Fatima passed MBBS from the medical college, appeared at the BCS examination, and finally passed. Now she is a doctor in a government hospital and serves the poor without fees.

\section*{2. The Golden Touch}

Long ago, in a kingdom filled with riches, there lived a king named Midas. King Midas was obsessed with gold. He spent his days counting his treasures and dreaming of ways to increase his wealth. His greed knew no bounds, and he believed that having more gold would bring him ultimate happiness.

One day, while wandering in the forest, Midas stumbled upon a mysterious stranger. The stranger offered to grant him one wish. Without hesitation, Midas exclaimed, “I wish that everything I touch turns into gold!” The stranger smiled mysteriously and granted his wish.

At first, Midas was overjoyed. Trees, stones, and even flowers turned into shining gold at his touch. He ran through the forest, turning everything he could find into precious metal. But soon, the consequences of his wish became apparent. When he sat down to eat, his food turned into gold, making it impossible for him to satisfy his hunger. Even worse, when he hugged his beloved daughter out of love, she too turned into a golden statue.

Heartbroken and desperate, Midas begged the stranger to take back the gift. The stranger agreed but warned him to be careful about what he desired in the future. From that day forward, Midas learned to value relationships and simple pleasures over material wealth.

This story reminds us that greed can lead to unhappiness, and true happiness comes from cherishing the people and moments in our lives rather than accumulating possessions.

\section*{3. Sheikh Saad’s Wit}
The king of Iran used to invite the great poet Sheikh Saadi very often to his court. Once on his way to the king’s court the poet took shelter in a noble man’s house for a night. He then took shelter in a nobleman’s house for a night. He was then in a very simple dress. The nobleman could not recognize him and treated him as an ordinary man because he was not wearing the costly and gorgeous dress. He was wearing a very ordinary and cheap cloth. The treatment he received in the rich man’s house offended him. But he did not say anything to him.

On the next day, Sheikh Saadi went to the court of the king and was received with honour. He stayed there for a few days. He composed some beautiful poems and entertained the king and his courtiers. He received rich gifts from the king. He put on a very rich and ornamented dress. Thus, he was returning to his village. On his way, he again visited that rich man’s house. This time, the rich man was surprised to see Sheikh Saadi in a rich ornamented dress. He was delighted to see the change in Sheikh Saadi and treated him with the best of the foods and comfort.

While eating, he was surprised to see the rich delicious food and understood the rich man’s attitude. Sheikh Saadi then instead of eating the food, started putting them into the pockets of his rich cloth. This surprised the rich man. “Why are you putting the food into the pockets of your dress?” said the rich man. “I am putting them into the pockets of my dress because these foods are meant for my dress, not for me”, said Saadi. The rich man realized his earlier folly and became ashamed.

\section*{4. Bread Dividing by a Cunning Monkey}
One day, two rats stole a piece of bread. They tried to divide it into two equal parts but failed. They fought over the matter. Finally, they agreed to take their problem to the monkey, who was known as the wisest animal in the forest.

They went to the monkey and asked him to divide the bread into two equal parts. The monkey behaved very gently with them. He said that it would be costly for them to have him resolve the issue and suggested that it would be better for them to settle it among themselves. But the rats replied that since they had failed to do so, they had come to him to find a solution to their problem.

Then the monkey brought a pair of scales and tore the bread into two pieces. He placed the pieces on the scales. The monkey declared that one piece was heavier than the other. He took the heavier piece and bit off a small portion from it. Then the other piece became heavier. The monkey then took a portion from that piece and placed it back on the scales. The pieces remained unequal even after this. The monkey continued doing the same thing repeatedly. After some time, only a small piece of bread remained.

At this point, the rats asked the monkey to stop dividing the bread. They said that they would divide it equally themselves and requested the remaining small piece of bread. The monkey angrily replied, “No, this piece is mine. I’ve suffered much to divide it into equal pieces.” Saying this, the monkey devoured the rest of the bread.

\section*{5. Who is to Bell the Cat?}
Once the house of a rich man was infested with rats. The house became like the town Hamelin. There were rats everywhere. They were having a good time. But the members could not enjoy sound sleep. Even the little babies were not free from the attack and biting of the rats.

The rats would take away foods, cut cloths and tear holes here and there. All the members were in great trouble. At last the owner of the house hit upon a plan. He knew that rats would run away at the sight of cats. So, he bought a cat. The mice were very afraid of the cat.

They fell in a great difficulty because they could not move freely as before. So, a few days later, all the mice held a meeting to discuss the matter and find a way to be free from this danger. Several proposals were made but none of the proposal was good enough to accept.

At one point, it seemed to them that they had to finish the meeting without finding any fruitful measure. Right at that moment, a young mouse rose to speak and said, “I have a good plan for your consideration. Let us tie a bell round the cat's neck.

Then we will hear him coming and be able to hide ourselves in time." All the mice appreciated the young mouse for his good plan and thanked him. At last, an old mouse stood up and said, “No doubt the idea of the young mouse! is good, But who will tie the bell?" Hearing this, all the mice remained silent and felt disappointed.

They understood it very well that it's really very dangerous to tie the bell round a cat's neck and he who will go to implement this task must die. At last the meeting ended without finding any way out and they had nothing to do but to migrate eventually.

\section*{6. Look Before You Leap}
Sufia is a worker in a big garments factory. More than five thousand workers work in that factory. One day, while she was busy at work, a sound was heard, ‘’Fire! Fire! Help! Help!’’. The shout pierced through the hum of sewing machines, sending a jolt of fear through Sufia and her coworkers. Instantly, the factory floor erupted into chaos as workers dropped their tools and rushed toward the nearest exit. No one paused to verify the source of the alarm; the word "fire" alone was enough to spark panic.

Sufia, caught in the wave of fleeing bodies, stumbled as she tried to keep up. The factory, despite its size, had only one narrow exit, barely wide enough for two people to pass through at once. As thousands surged toward it, the passageway clogged with desperate workers, their shouts blending into a deafening roar. Sufia saw a woman beside her fall, trampled underfoot before she could be helped. Her heart raced as she realized the danger wasn’t just from a potential fire, but from the crowd itself.

Pushing forward, Sufia managed to reach a corner where she could catch her breath. She glanced back, horrified to see injured workers sprawled across the floor, some clutching broken limbs. The air was thick with dust and the sharp smell of fear. Just then, a crackling voice boomed over the loudspeakers: “Attention! There is no fire. This is a false alarm. Please return to your stations.” The announcement repeated, cutting through the hysteria.

Sufia froze, her mind racing as the truth sank in. She learned later that a pipe in the dyeing section had burst, spewing colored liquid that someone mistook for smoke. The workers, including herself, had acted without thinking, and the result was chaos instead of safety. Slowly, the crowd thinned as people returned to their posts, some limping, others whispering apologies. Sufia sat back at her machine, her hands trembling as she resumed stitching.

That day taught her a hard lesson: panic could be as deadly as any flame. She vowed to never again act without confirming the truth, no matter how loud the cries around her. The factory resumed its rhythm, but the scars of that moment lingered, a silent reminder to look before leaping into action.

\section*{7. A Scholar and a Boatman}
There was a poor boatman in a village. He was illiterate. He used to row boat from morning till evening only to meet his both ends meet. It was the beginning of the summer season. There was only one passenger in the boat who was a scholar. The boatman set sail and the boat was advancing smoothly.

The scholar said to the boatman, “Did you read history?” The boatman said, “No.” The scholar told him that without any knowledge of history one-fourth of his life was spoilt. He looked at the beautiful scenery and asked again whether he had read geography.

As usual the reply of the boatman was in the negative. This time the scholar said that one half of the boatman's life was spoilt. Then they were silent for some time.

It was afternoon. The scholar broke the silence and said to the boatman, “Do you know anything about science?” He replied in the negative. The scholar said to him that his life was of no use. Three-fourths of his life were spoilt. Hearing this the boatman remained silent.

All on a sudden the sky became covered with dark clouds. The Nor'wester began to blow in the form of a storm. The scholar was very much frightened. This time the boatman said to him, “Sir, do you know how to swim?”

The scholar replied in the negative in a pitiful voice. The boatman said, “The boat is going to sink. Now I see, the whole of your life is spoiled. Your bookish knowledge is of no use.”

\section*{8. Responsibility of a Young Boy}
Once a boy named Arif was coming home from his school on foot. Suddenly he noticed a moneybag lying beside the road. He thought for a while. He was confused to take up the bag as somebody might treat him as a thief. Again he was not such irresponsible as to overlook the matter. So, he shook off hesitation and took the bag to his house.

At home, he consulted the fact with his parents and his younger sister. His parents advised him to search for the owner of the bag and he found a big amount of money. Inside the moneybag, he found a visiting card with phone number. Arif decided to call on the number. A grave male voice answered on the other end. Arif enclosed his identity and the fact of getting the moneybag. The receiver demanded that he owned that bag with money. Arif tried to test the man if he was speaking the truth. The conversation between them removed Arif's confusion. Arif became sure of his ownership of the bag.

He gave his residence's address to the man. The man met Arif and thanked him a lot for his responsibility. He also gifted Arif a nice story book.

\section*{9. Honesty of a Rickshaw Puller}
Once there was a poor rickshaw-puller whose name was Abdur Rahman. Though he was very poor, he was honest. So, he earned his livelihood at the cost of his honest sweat . He never liked to be a rich man over night. One day while Raju was returning home with his rickshaw, he found a small leather bag on his rickshaw. He at once could understand that the bag must be dropped from the passenger he just had dropped at New Market. He opened the bag and found a lot of money inside it. He thought that he had to give back the bag to its original owner. But how? He did not find a way-out . After a few moments he remembered that he took the passenger on his rickshaw in front of a house near Dhanmondi. “May he the passenger belongs to that house,” he thought.

Thinking so, he came to that particular house. He met the housewife and told her about the matter. The house wife knew that her husband had lost a leather bag that was in black colour. Then the rickshaw puller showed her the leather bag. It was the real bag. The housewife could instantly identify the bag as her husband’s. She phoned her husband telling everything. He came home within a very short time. The rickshaw puller could easily identify the owner of the bag and gladly returned him his possession . The owner and his wife thanked him very much and offered some money. But the rickshaw-puller politely refused the offer. The looks of gratitude in the eyes of the rich couple were his reward indeed.
\section*{10. The Magnanimity of Gias Uddin Azam Shah}
Giashuddin Azam Shah was the ruler of Bengal, known for his kindness and sense of justice. He was also very fond of archery. One day, while practicing archery, his arrow accidentally missed its target and struck a young boy, killing him instantly. The boy was the only son of a widow, who was devastated by the loss. Seeking justice, the widow went to the Qazi (judge) and filed a complaint against the Sultan himself.

The Qazi, upholding the law impartially, summoned the Sultan and said, “According to the law of the land, you must face punishment for this accident. However, if you can compensate the widow and gain her satisfaction, you may avoid the punishment.” The Sultan, recognizing the fairness of the ruling, willingly provided generous compensation to the widow, who was then satisfied with the resolution.

After the matter was settled, the Qazi stepped down from his seat and showed the Sultan due respect for accepting the judgment humbly.

However, the Sultan then drew his sword and said, “Listen carefully, O Qazi. If you had failed to deliver justice to the widow, I would have cut off your head with this sword.” In response, the Qazi calmly drew out his cane and replied, “And if you, O Sultan, had disobeyed my ruling, I would have beaten you with this cane.”

Impressed by the Qazi’s courage and unwavering commitment to justice, the Sultan thanked him warmly and embraced him as a sign of respect

\end{document}